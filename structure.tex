%%%%%%%%%%%%%%%%%%%%%%%%%%%%%%%%%%%%%%%%%
% The Szabo Book
% Structural Definitions File
% Version 0.1 (5/4/20)
%
\RequirePackage{luatex85,shellesc}
\usepackage[top=1.7cm,bottom=1.5cm,left=1.7cm,right=1.7cm,headsep=18pt,papersize={13.8176cm,21.6cm}
]{geometry}
\usepackage{indentfirst} 
\usepackage{graphicx} % Required for including pictures
\graphicspath{{Pictures/}}
\usepackage{multirow}
\usepackage{tikz,pgfplots} % Required for drawing custom shapes
%\pgfplotsset{compat=newest}
\usepackage{booktabs,threeparttable
} % Booktabs required for nicer horizontal rules in tables
\usepackage[shortlabels]{enumitem}
\setenumerate{leftmargin=*}
\usepackage{xcolor} % Required for specifying colors by name
%\definecolor{ocre}{RGB}{0,0,0} % Define the favorite color used for highlighting throughout the book
\bibliographystyle{alpha}


%-------------------------------------------------------------------
%	Chinese Settings
%-------------------------------------------------------------------
\usepackage[format=hang,labelfont=bf]{caption}
%\captionsetup{font={scriptsize}}
\captionsetup{figurename={图}, tablename={表}}
%\addto{\captionsenglish}{\renewcommand{\contentsname}{目\quad录}}
\def\chntoday{\the\year~年~\the\month~月~\the\day~日}
% adjust the width of tables to adapt it to the textwidth
\usepackage{tabularx}
\newcolumntype{Y}{>{\centering\arraybackslash}X} % New flag to centering the width-adapted columns

\usepackage{amsmath}
\usepackage{anyfontsize}
\usepackage{amssymb}
\usepackage{bm}

\usepackage{mathtools} % For *rcases* environment
\usepackage[scr=boondoxo, scrscaled=1]{mathalfa}
\usepackage{mathrsfs}
%\usepackage{mathptmx}
%\usepackage{mathpazo}
%\usepackage{old-arrows} % 
%\usepackage[T1]{fontenc}
%\usepackage{newpxtext,newpxmath}
%\usepackage{newtxtext}
%\usepackage{newtxmath,} %A systematic solution to Roman fonts, including math fonts.
\usepackage{braket,ulem}
\renewcommand{\braket}[1]{\Braket{#1}}
\renewcommand{\ket}[1]{\Ket{#1}}
\renewcommand{\bra}[1]{\Bra{#1}}
\ctexset{
  chapter={
  	name = {第,章},
  	format = \Large\bfseries\raggedright,
  	aftername = \par\bigskip},
  section/format = \fontsize{8.5pt}{\baselineskip}\selectfont\bfseries\raggedright,
  subsection/format = \fontsize{8pt}{\baselineskip}\selectfont\bfseries\raggedright
}

%-----------------------------------------------------------------
%	Page Headers
%-----------------------------------------------------------------

\usepackage{fancyhdr} % Required for header and footer configuration
\renewcommand{\chaptermark}[1]{\markboth{\sffamily\normalsize\CTEXthechapter\ #1}{}} % Chapter text font settings
\renewcommand{\sectionmark}[1]{\markright{\normalfont\kaishu\normalsize\CTEXthesection\hspace{5pt}#1}{}}

%----------------------------------------------------------------------------------------
\let\bar\undifined
\newcommand{\bar}[1]{\overline{#1}}
\newcommand{\scr}[1]{\mathscr{#1}}
\newcommand{\bo}[1]{\mathbf{#1}}
\renewcommand{\epsilon}{\varepsilon}
\newcommand{\sch}{Schr\"odinger}
\newcommand{\ha}{Hamiltonian}
%\newcommand{\dd}{\text{d}}
\usepackage{xparse}
\newcommand{\dd}[1]{\mathrm{d}\mathbf{#1}}
\let\dd\undefined
\NewDocumentCommand{\dd}{ g }{%
	\IfValueTF{#1}
	{% https://tex.stackexchange.com/q/53068/5764
		\if\relax\detokenize{#1}\relax
		\mathrm{d} %
		\else
		\mathrm{d}\mathbf{#1}%
		\fi}
	{\mathrm{d}}%
}
\newcommand{\ddx}{\dd\mathbf{x}}

\newcommand{\db}[1]{\dd\bo{#1}}
\newcommand{\hs}{\mathscr{H}}
\newcommand{\vs}{\mathscr{V}}
\newcommand{\es}{\mathscr{E}}
\renewcommand{\emph}[1]{\textbf{#1}}


\usepackage{calc}
%use \settowidth, \widthof

\usepackage{endnotes}
\renewcommand{\makeenmark}{\hbox{$^\theenmark$}}
\renewcommand{\notesname}{\bf 注释}
\makeatletter
\@addtoreset{endnote}{chapter}
\makeatother

%-----------------------------------------------------------------
% the command to generate two lines between which are excerise content, using amsthm to number the exercise.
\let\openbox\undefined
\usepackage{amsthm}
\usepackage{needspace}
\newtheoremstyle{nx}{}{}{\normalfont}{}{\bfseries}{}{1em}{}
\theoremstyle{nx}
\newtheorem{Xercise}{$\quad$练习}[chapter]
\newenvironment{xercise}{\Needspace{2\baselineskip}\par\vspace{\baselineskip}\noindent\hrule\vspace{0\baselineskip}\Xercise\label{ex:\theXercise}
}
{\vspace{.5\baselineskip}\noindent\hrule\par\endXercise}

\newcommand\Next{\endXercise\hrule\Xercise\label{ex:\theXercise}}
\newcommand{\exercise}[1]{\begin{xercise}#1\end{xercise}}
%----------------------------------------------------------------------------------------

% 引用章节、方程、图、表时自动写成“第……章(节)”、“式(x.y)”、“图x.y”、“表x.y”这样的型式:
\def\chapterautorefname~#1\null{第#1章\null}
\def\sectionautorefname~#1\null{小节#1\null}
\def\subsectionautorefname~#1\null{小节#1\null}
\def\subsubsectionautorefname~#1\null{小节#1\null}
\def\equationautorefname~#1\null{式(#1)\null}
\def\tableautorefname~#1\null{表#1\null}
\def\figureautorefname~#1\null{图#1\null}
\def\appendixautorefname~#1\null{附录#1\null}
\def\Xerciseautorefname~#1\null{练习#1\null}

%========================================================

\newcommand{\tu}{\begin{figure}[h]
		\includegraphics[width=.5\textwidth]{q.png}
\end{figure}}
% Feynman Diagram Setting
\usetikzlibrary{arrows.meta}
\newcommand{\midarrow}{\tikz \draw[-triangle 45] (0,0) -- +(.05,0);}
\tikzset{> = latex, color=blue}
\usetikzlibrary{decorations.markings,decorations.pathreplacing}
\tikzset{
	% style to apply some styles to each segment of a path
	on each segment/.style={draw=blue,
		decorate,
		    decoration={
			show path construction,
			moveto code={},
			lineto code={
				\path [#1]
				(\tikzinputsegmentfirst) -- (\tikzinputsegmentlast);
			},
			curveto code={
				\path [#1] (\tikzinputsegmentfirst)
				.. controls
				(\tikzinputsegmentsupporta) and (\tikzinputsegmentsupportb)
				..
				(\tikzinputsegmentlast);
			},
			closepath code={
				\path [#1]
				(\tikzinputsegmentfirst) -- (\tikzinputsegmentlast);
			},
		},
	},
	% style to add an arrow in the middle of a path
	mid arrow/.style={postaction={decorate,decoration={
				markings,
				mark=at position #1 with {\draw[arrows = {-Latex[width=0pt 7, length=5pt]}] (0pt,.0pt) -- (1.8pt,0pt);}
	}}},
	mid arrow/.default={.55},
	mid arrow seg/.style={postaction={on each segment={decorate,decoration={
				markings,
				mark=at position #1 with {\draw[arrows = {-Latex[width=0pt 7, length=5pt]}] (0pt,.0pt) -- (1.8pt,0pt);}}
	}}},
	mid arrow seg/.default={.55},
	% style to add an reverse arrow in the middle of a path
	mid reverse arrow/.style={postaction={decorate,decoration={
			markings,
			mark=at position .45 with {\draw[arrows = {-Latex[width=0pt 7, length=5pt]}] (1.8pt,.0pt) -- (0pt,0pt);}
	}}},
	mid reverse arrow seg/.style={postaction={on each segment={decorate,decoration={
				markings,
				mark=at position #1 with {\draw[arrows = {-Latex[width=0pt 7, length=5pt]}] (1.8pt,.0pt) -- (0pt,0pt);}}
	}}},
	above arrow/.style={to path={-- ++(0, .2) -| (\tikztotarget)},arrows = {-Latex[width=0pt 7, length=5pt]}},
	below arrow/.style={to path={-- ++(0,-.2) -| (\tikztotarget)}}
}
\newlength{\wletter}
\newlength{\hletter}
\newcommand{\tikzmark}[1]{
	\settowidth{\wletter}{\text{$\mathsurround=0pt i$}}
	\settoheight{\hletter}{\text{$\mathsurround=0pt i$}}
	\tikz[overlay,remember picture,inner sep=0,minimum height=\hletter] \node[shift={(-.4\wletter,.45\hletter)}] (#1) {};}
\newcommand{\dian}{\tikz{\filldraw[blue](0,0)circle(1pt);}}

\usetikzlibrary{calc}
%\usepackage[compat=1.1.0]{tikz-feynman}
\usepackage{float}
\renewcommand{\thefootnote}{\roman{footnote}} % Change the marks of footnotes to numbers
\usepackage{blkarray} %分块矩阵
\usepackage[bookmarksnumbered=true]{hyperref}
\usepackage{chemformula}


\usepackage{wasysym}
%For the symbol \CheckedBox
\usepackage{stackengine,scalerel}
\newcommand\DSquare{\ThisStyle{\ensurestackMath{%
			\stackinset{c}{}{c}{}{\scalebox{.5}{$\SavedStyle\blacksquare$}}
			{\SavedStyle\square}}}}
%For a filled box.
%CR: https://tex.stackexchange.com/questions/553834/is-there-a-intermediate-checkedbox-symbol-in-latex

%-----------------------------------------------------------------
%	Convinent abbreviations
%-----------------------------------------------------------------
\renewcommand{\normalsize}{\fontsize{8pt}{9.6pt}\selectfont}
\newcommand{\diagsize}{\fontsize{5pt}{6.6pt}\selectfont}
\newcommand{\cs}{a^\dagger}
\newcommand{\hd}{\mathrm{H}_2}
\newcommand{\hft}{Hartree-Fock}
\newcommand{\twoe}{r_{12}^{-1}}
\newcommand{\tp}{\tilde{\Phi}}
%\usepackage[symbol]{footmisc} 
\newcommand{\bfr}{\mathbf{r}}
\newcommand{\heh}{\mathrm{HeH}^+}
\newcommand{\au}{\,\mathrm{a.u.}}
\newcommand{\ts}{\mathscr{S}} %总自旋算符
\usepackage{xeCJKfntef} % the package containing the command \CJKuderline*{} for the \mci command
\newcommand{\mci}[1]{\CJKunderline{#1}}
\newcommand{\phrase}[1]{\CJKunderline{#1}}
%%为方便而定义的简单代替, 最后要全部换回来.
\newcommand{\jt}{\ket{\Phi_0}}
\newcommand{\hjt}{\ket{\Psi_0}}
\newcommand{\jtn}{\Phi_0}
\newcommand{\hjtn}{\Psi_0}

%Convenience commands defined by Hao
\newcommand{\mrm}{\mathrm}
\newcommand{\mbf}{\mathbf}
\newcommand{\mcr}{\mathscr}
\newcommand{\op}{\mathscr}
\newcommand{\bs}[2]{\ensuremath\left\{ \vec{#1}_{#2} \right\}}
\newcommand{\kbs}[1]{\ensuremath\left\{ \ket{#1} \right\}}
\newcommand{\bbs}[1]{\ensuremath\left\{ \bra{#1} \right\}}
\newcommand{\madj}[1]{\ensuremath{\mbf #1}^\dagger}
\newcommand{\oadj}[1]{\ensuremath{\op #1}^\dagger}
\newcommand{\tr}[1]{\ensuremath{\mrm{tr}}\,#1}
%\newcommand{\ket}[1]{\ensuremath\left\vert #1\right\rangle}
%\newcommand{\bra}[1]{\ensuremath\left\langle #1\right\vert}
\newcommand{\Langle}{\ensuremath\Big\langle}
\newcommand{\Rangle}{\ensuremath\Big\rangle}
\newcommand{\Lvert}{\ensuremath\Big\vert}
\newcommand{\olp}[2]{\ensuremath\langle #1 \vert #2 \rangle}
\newcommand{\Olp}[2]{\ensuremath\left\langle #1 \middle\vert #2 \right\rangle}
\newcommand{\oup}[2]{\ensuremath\left\vert #1 \right\rangle \left\langle #2 \right\vert}
\newcommand{\mele}[3]{\ensuremath\langle #1\vert \op{#2} \vert #3\rangle}
\newcommand{\Mele}[3]{\ensuremath\left\langle #1\middle\vert \op{#2} \middle\vert #3\right\rangle}
\newcommand\bigzero{\makebox(0,0){\text{\huge0}}}
%=============END===================

