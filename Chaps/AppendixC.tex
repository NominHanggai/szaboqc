\chapter{几何结构优化与解析导数法(作者:M.C.Zerner)}
Michael C.Zerner

量子理论项目

佛罗里达大学
\section{引言}

分子构象的再现和预测是分子量子力学的最成功的应用之一。
许多情况下,对于很多简单的分子轨道模型或者最小基组的从头算方法,键长可以重现到$\pm 0.02$\r A,
键角可以重现到$5^{\circ}$。
对于大一点的基组,尤其是那些双$\zeta$加极化的基组;和包含电子相关能的方法,现在生成的几何构型已经
可以达到晶体学精度了。
随着构象计算的日益成功,人们甚至可以选择孤立分子的计算结果,而不是在凝聚介质中获得的实验结果,因为前者可能更适合气相。

除了产生关于势能面全局最小点的信息外,量子力学计算还可以产生局域最小点的信息,这些局域最小点可能不会被直接观测到,但是
很可能会被包哦旱灾反应路径中。类似的,关于过渡态和能垒的信息也可以得到,这些信息一般很难甚至不可能通过其他方式获得。

收集一个势能面上这些所有的信息是很困难的。对于N个原子的体系,它的能量是一个具有$3N-6$(或者$3N-5$)个自由度的函数。
为了进行详细的统计计算,人们可能不得不面对这个“3N”问题,并访问势能面上所有热力学可得的区域。
然而,本附录只涉及势能面的一小部分:那些对应于极小值的点,代表稳态或亚稳态的构象,以及对应于过渡态的点。
%%%%%%%
\section{概论}

在Born-Oppenheimer近似下得到的分子体系的能量$E$是一个以核坐标为参数的函数,记核坐标为
$\mathbf{X}^{\dagger}=(X_1,X_2,\dots,X_{3N})$。
我们希望从$E(\mathbf{X})$进而得到$E(\mathbf{X_1})$,$\mathbf{q=(X_1-X)}$。
我们将能量对$\mathbf{X}$进行泰勒展开:
\begin{align}
	\label{C.1}
	E(\mathbf{X_1})=E(\mathbf{X})+\mathbf{q}^{\dagger}\mathbf{f(X)}
                    +\frac{1}{2}\mathbf{q}^{\dagger}\mathbf{H(X)}\mathbf{q}+\dots
\end{align}
式中梯度为
\begin{align}
	f_i=\frac{\partial E(\mathbf{X})}{\partial X_i}
    \nonumber
\end{align}
Hessian矩阵元为
\begin{align}
	H_{ij}=\frac{\partial E(\mathbf{X})}{\partial X_i\partial X_j}
    \nonumber
\end{align}
注意,列矩阵的下标表示不同的矩阵,如$\mathbf{X_1}$、$\mathbf{X_2}$等等;而$X_i$表示矩阵$\mathbf{X}$第i个元素。
虽然泰勒展开是无穷项的,但是接近极值时,我们希望二阶展开是足够的;例如,对于$\mathbf{X}=\mathbf{X_e}$,其中$\mathbf{X_e}$
表示一个驻点,根据定义$\mathbf{f(X_e)=0}$,则
\begin{align}
	\nonumber
	E(\mathbf{X_1})=E(\mathbf{X_e})+\frac{1}{2}\mathbf{q}^{\dagger}\mathbf{H(X_e)}\mathbf{q}
\end{align}
类似的,
\begin{align}
	\label{C.2}
	\mathbf{f(X_1)}=\mathbf{f(X)}+\mathbf{H(X)}\mathbf{q}
\end{align}
对点$\mathbf{X_1}=\mathbf{X_e}$
\begin{align}
	\label{C.3}
	\mathbf{f(X)}=-\mathbf{H(X)}\mathbf{q}
\end{align}

\autoref{C.3}的解是不显含$\mathbf{X}$的$E(\mathbf{X})$泛函寻找多变量函数极值的最高效方法的初始点。
如果$\mathbf{H}$是非奇异的,则有
\begin{align}
	\label{C.4}
	\mathbf{q}=-\mathbf{H^{-1}(X)}\mathbf{f(X)}
\end{align}
这能够从任意一点$\mathbf{X}$解得$\mathbf{X_e}$,从而使能量函数接近二阶展开。同样的,一个预测的能量$\mathbf{E(H_e)}$
\footnote{译者注:原文如此,此处可能是笔误,结合上下文来看,应为$\mathbf{E(X_e)}$}
可以从下式得到
\begin{align}
	\label{C.5}
	\mathbf{E(X_e)}&=\mathbf{E(X)}-\frac{1}{2}\mathbf{f(X)}^{\dagger}\mathbf{H^{-1}(X)}\mathbf{f(X)}
	\nonumber\\
	&=\mathbf{E(X)}-\frac{1}{2}\mathbf{q}^{\dagger}\mathbf{H(X_e)}\mathbf{q}
\end{align}
对于特殊的势能面寻找极值的问题,我们必须要指出:除非将代表$\mathbf{H}$的零特征值的旋转和平移移除,
否则$\mathbf{H^{-1}(X)}$将不存在。这个工作可以通过Wilson和Eliashevich提出的$\mathbf{B}$矩阵实现:
\endnote{See,for example,E.B.Wilson,J.C.Decius,and P.C.Cross,$Molecular Vibrations$,
McGraw-Hill,New York,1955.}
\begin{align}
	\label{C.6}
	\mathbf{Y}^{\dagger}=\mathbf{X}^{\dagger}\mathbf{B}
\end{align}
式中$\mathbf{X}$是$3N$维,$\mathbf{B}$是$3N\times 3(N-6)$维,关联内坐标和原本的笛卡尔坐标$\mathbf{X}$。
揭示这些新坐标下的Taylor级数结构的最简单的方法是考虑功$w$,因为它与坐标系的选择是独立的。
功是力和距离的乘积,能够在任何坐标系下表示为
\begin{align}
	\nonumber
	w=\mathbf{f}^{\dagger}\mathbf{q}=\mathbf{f_y}^{\dagger}\mathbf{q_y}=\mathbf{f_y}^{\dagger}\mathbf{B}^{\dagger}\mathbf{q}
\end{align}
\begin{align}
	\nonumber
	\mathbf{f}^{\dagger}=\mathbf{f_y}^{\dagger}\mathbf{B}^{\dagger}
\end{align}
或者
\begin{align}
	\label{C.7}
	\mathbf{f_y}=\mathbf{f}^{\dagger}(\mathbf{B}^{\dagger})^{-1}
\end{align}
式中$(\mathbf{B}^{\dagger})^{-1}$满足
\begin{align}
	\nonumber
	\mathbf{B}^{\dagger}(\mathbf{B}^{\dagger})^{-1}=\mathbf{1}
\end{align}
下面给出上述方程的一般解为
\begin{align}
	\nonumber
	(\mathbf{B}^{\dagger})^{-1}=\mathbf{mB}(\mathbf{B^{\dagger}mB})^{-1}
\end{align}
式中$\mathbf{m}$是一个任意的$3N\times 3N$矩阵,经常选择一个对角矩阵,
矩阵元素是每一个对应位置上原子的原子质量的倒数的三倍。
也可以选择一个单位矩阵,其中6(或5)个元素选择为0去阻止平动和转动。
这种类型一个简单的选择是将原子1放在原点,原子2在z轴上,原子3在xz平面上。
于是这被移除的6(或5)个坐标为$x_1=y_1=z_1=0$,$x_2=y_2=0$,$y_3=0$。
如果$y_3=0$对于任意选择的第三个原子来说都意味着$x_3=0$,那么这个分子是线性的并且只有5个自由度被选择。
实际操作中,求$\mathbf{H}$的逆矩阵被证明有一点困难。
接下来要讨论的更新的方法直接构建$\mathbf{H}^{-1}$,并且从不更新平动和转动。
在解析求解$\mathbf{H}$的时候,平动和转动可以像之前讨论的那样被移除,或者,如果通过对角化求逆的话,
$\mathbf{H}$的6(或5)个为0的特征值会被一个任意的大数替换掉,本质上在求逆中解耦这些状态。
%%%%%
\section{解析导数}
我们现在来考虑量子化学中$\mathbf{f}$和$\mathbf{H}$的计算。一个直接的方式是简单的数值微分能量。
取而代之的是,我们可以通过解析求导的方式得到这些导数。让我们来举例说明Hartree-Fock计算框架中的
一些基本思想。

Hartree-Fock能量显示的依赖占据轨道系数$\mathbf{C}$和$\mathbf{X}$。它的导数由下式给出
\begin{align}
	\label{C.8}
	\frac{\partial E}{\partial X_A}=\frac{\partial {\tilde{E} }}{\partial X_A}
	+\sum_{\mu a}\frac{\partial E}{\partial C_{\mu a}}\frac{\partial C_{\mu a}}{\partial X_A}
\end{align}
式中,$\frac{\partial {\tilde{E} }}{\partial X_A}$表示所有显式依赖核坐标$X_A$的项的导数,
并且其中链式规则项源于分子轨道系数对几何结构的隐含依赖性。因为$\frac{\partial E}{\partial C_{\mu a}}=0$
是Hartree-Fock解的条件(\autoref{sec3.2.1}),
\begin{align}
	\label{C.9}
	\frac{\partial E}{\partial X_A}=\frac{\partial {\tilde{E} }}{\partial X_A}
\end{align}
这种实现允许我们忽略分子轨道系数相对于几何结构变化的一阶变化。
一个闭壳层体系在Hartree-Fock近似下的总能量可以表示为
\begin{align}
	\label{C.10}
	E=\sum_{\mu \nu}P_{\nu \mu}H_{\mu \nu}^{core}
		+\frac{1}{2}\sum_{\mu \nu \lambda \sigma }P_{\nu \mu}P_{\lambda \sigma}(\mu \nu||\sigma \lambda)
		+V_{NN} 
\end{align}
(见\autoref{3.184},\autoref{3.185}和\autoref{3.154})式中我们已经定义
\begin{align}
	\nonumber
	V_{NN}=\sum_{A}\sum_{A \ge B}\frac{Z_A Z_B}{R_{AB}}
\end{align}
以及介绍了速记符号
\begin{align}
	\nonumber
	(\mu \nu||\sigma \lambda)=(\mu \nu|\sigma \lambda)-\frac{1}{2}(\mu \lambda ||\sigma  \nu)
\end{align}
对于实轨道,密度矩阵表示为
\begin{align}
	\nonumber
	P_{\nu \mu}=2\sum_{a}^{N/2}C_{\mu a}C_{\nu a}
\end{align}
(见\autoref{3.145})。微分\autoref{C.10}得
\begin{align}
	\label{C.11}
    \frac{\partial E}{\partial X_A}=
    &\sum_{\mu \nu}P_{\nu \mu}\frac{\partial H_{\mu \nu}^{core}}{\partial X_A}
    +\frac{1}{2}\sum_{\mu \nu\lambda \sigma }P_{\nu \mu}P_{\lambda \sigma}
      \frac{\partial (\mu \nu|| \sigma\lambda)}{\partial X_A}
    +\frac{\partial V_{NN}}{\partial X_A} 
	\\ \nonumber
	&+\sum_{\mu \nu}\frac{\partial P_{\nu \mu}}{\partial X_A} H_{\mu \nu}^{core}
	+\sum_{\mu \nu\lambda \sigma }\frac{\partial P_{\nu \mu} }{\partial X_A}P_{\lambda \sigma}(\mu \nu|| \sigma\lambda)
\end{align}
\autoref{C.11}表明需要求取组合系数的导数,然而\autoref{C.9}不需要!
展开\autoref{C.11}的后两项得到
\begin{align}
	\nonumber
	=&4\sum_{\mu \nu}\sum_{a}^{N/2}\frac{\partial C_{\mu a}}{\partial X_A}H_{\mu \nu}^{core}C_{\nu a}
		+4\sum_{\mu \nu\lambda \sigma }\sum_{a}^{N/2}\frac{\partial C_{\mu a}}{\partial X_A}  P_{\lambda \sigma}(\mu \nu|| \sigma\lambda)  C_{\nu a}
	\\ \nonumber
	=&4\sum_{\mu \nu}\sum_{a}^{N/2}\frac{\partial C_{\mu a}}{\partial X_A}
		[H_{\mu \nu}^{core}-\sum_{\lambda \sigma } P_{\lambda \sigma}(\mu \nu|| \sigma\lambda)]C_{\nu a}
	\\ \nonumber
	=&4\sum_{\mu \nu}\sum_{a}^{N/2}\frac{\partial C_{\mu a}}{\partial X_A}F_{\mu \nu}C_{\nu a}
	\\ \nonumber
	=&4\sum_{a}^{N/2}	\varepsilon_a \sum_{\mu \nu}\frac{\partial C_{\mu a}}{\partial X_A}S_{\mu \nu}C_{\nu a}
\end{align}
为了计算系数的导数,我们回顾一下分子轨道的正交归一化条件,即
\begin{align}
	\nonumber
	\sum_{\mu \nu}C_{\mu a}S_{\mu \nu}C_{\nu a}=\delta_{ab} 
\end{align}
(见练习3.10)。微分上式得到
\begin{align}
	\nonumber
	2\sum_{\mu \nu}\frac{\partial C_{\mu a}}{\partial X_A}S_{\mu \nu}C_{\nu a}
	=-\sum_{\mu \nu}C_{\mu a}C_{\nu a}\frac{\partial S_{\mu \nu}}{\partial X_A}
\end{align}
组合这些表达式的结果得
\begin{align}
	\label{C.12}
    \frac{\partial E}{\partial X_A}=
    \sum_{\mu \nu}P_{\nu \mu}\frac{\partial H_{\mu \nu}^{core}}{\partial X_A}
    +\frac{1}{2}\sum_{\mu \nu\lambda \sigma }P_{\nu \mu}P_{\lambda \sigma}
      \frac{\partial (\mu \nu|| \sigma\lambda)}{\partial X_A}
    -\sum_{\mu \nu}Q_{\nu \mu} \frac{\partial S_{\mu \nu}}{\partial X_A}
    +\frac{\partial V_{NN}}{\partial X_A}
\end{align}
式中我们定义
\begin{align}
	\nonumber
    Q_{\nu \mu}=2\sum_{a}^{N/2}\varepsilon_a C_{\mu a}C_{\nu a}
\end{align}
因此能量的导数可以通过分子轨道系数和重叠积分以及单电子、双电子积分的导数计算得到。

以高效的方式获取电子结构理论中出现的积分的导数是一个有点专业化的领域,但我们很容易理解其一般思路。
许多从头计算都是使用笛卡尔高斯函数进行的
\begin{align}
	\nonumber
    \phi_{lmn}^{GF}=N_{lmn}x_{a}^{l}y_{a}^{m}z_{a}^{n} e^{-\alpha|\mathbf{r}-\mathbf{R_{A}}|^2} 
\end{align}
见\autoref{sec:3.6},式中$N_{lmn}$是归一化参数,表达式为
\begin{align}
	\nonumber
    N_{lmn}=\bigg[\frac{(8\alpha)^{l+m+n}l!m!n!}{(2l)!(2m)!(2n)!} \bigg]^{1/2}(\frac{2\alpha}{\pi})^{3/4}
\end{align}
其中小写字母a表示从原子核A测量的电子坐标。$\phi_{lmn}^{GF}$对核坐标的导数是可以显式直接得到的,例如
\begin{align}
	\nonumber
    \frac{\partial \phi_{lmn}^{GF}}{\partial X_A}=[(2l+1)a]^{1/2}\phi_{l+1.mn}^{GF}-2l(\frac{a}{2l-1})^{1/2}\phi_{l-1.mn}^{GF}
\end{align}
式中,现在的$X_A$特指核A的x坐标。需要指出的是当$l=0$时,不考虑上式第二项。

所有单中心积分的导数为零,因为通常假设中心A上的所有轨道都遵循中心A的位移。动能算符和电子排斥势算符$r_{12}^{-1}$不是核坐标的函数。
核排斥势能$V_{NN}$的导数也可以直接得到
\begin{align}
	\nonumber
    \frac{\partial V_{NN}}{\partial X_A}=Z_A\sum_{B}\frac{Z_B(X_B-X_A)}{R_{AB}^{3}}
\end{align}
对于核-电子吸引项,可以得到
\begin{align}
	\nonumber
    \frac{\partial V_{Ne}}{\partial X_A}=-Z_A\sum_{i}\frac{X_i-X_A}{r_{iA}^{3}}
\end{align}
上面的公式已经足够去求解电子结构计算中遇到的电子积分的导数了。但在实际计算中,需要使用许多技巧去减少计算量。
使用Slater函数去计算梯度会更困难(见\autoref{sec:3.5}和\autoref{sec:3.6}),但是至少对于经常在半经验模型中出现的
双中心积分而言,可以沿着与高斯函数相同的方式进行推导。

\autoref{C.12}相对简单,因为$\mathbf{P}$的导数没有出现,这不应与Hellmann-Feynman定理
\endnote{H.Hellmann,$Einf\ddot{u}hrung in du Quantenchmie$,Franz Deuticke,Leipzig,1937;
R.P.Feynman $ Phys.Rev.\bf{41}:$721(1939);
A.C.Hurley,$Proc.Roy.Soc.\bf{A226:}$170,179(1954)}
混淆。现在有
\begin{align}
	\nonumber
    E= \braket{\Phi |\mathcal{H} |\Phi }
\end{align}
且$\braket{\Phi |\Phi }=1$,则
\begin{align}
	\label{C.13}
    \frac{\partial E}{\partial X_A}=\braket{\frac{\partial \Phi}{\partial X_A} |\mathcal{H} |\Phi }+
									 \braket{\Phi|\mathcal{H} | \frac{\partial\Phi}{\partial X_A} }+
									 \braket{\Phi|\frac{\partial \mathcal{H}}{\partial X_A} | \Phi }
\end{align}
Hellmann-Feynman条件如下
\begin{align}
	\label{C.14}
    \braket{\frac{\partial \Phi}{\partial X_A} |\mathcal{H} |\Phi }+ \braket{\Phi|\mathcal{H} | \frac{\partial\Phi}{\partial X_A} }=0
\end{align}
上式只在精确解或者特定类型的试探函数的时候才成立。在\autoref{C.14}的约束下,\autoref{C.13}简化为
\begin{align}
	\label{C.15}
    \frac{\partial E}{\partial X_A}=\braket{\Phi|\frac{\partial \mathcal{H}}{\partial X_A} | \Phi }
\end{align}
\autoref{C.15}是一个简单的单电子算符的期望值加上核排斥项的导数项。但是\autoref{C.12}并不依赖\autoref{C.14}。
$\frac{\partial H^{core}}{\partial X_A}$和$ \frac{\partial (\mu \nu|| \sigma\lambda)}{\partial X_A}$中的积分
通过“原子轨道跟随”包含了波函数——例如,$\frac{\partial \phi_{\mu}^{A} }{\partial X_A}$,其中$\phi_{\mu}^{A}$
是一个以$A$为中心的原子轨道——并且比\autoref{C.15}复杂的多。事实上,即便在\autoref{C.13}下受力应该为0,
但通过\autoref{C.15}算出来的受力也会很大,因此代表了能量函数的极值。然而,\autoref{C.15}的简洁很有吸引力,
人们想知道,当目标是几何优化时,满足\autoref{C.14}的条件所增加的不便是否不像在使用\autoref{C.15}那样快。

对于一个组态相互作用(CI)波函数,含有行列式$\ket{\Psi _I}$,
\begin{align}
	\nonumber
    \ket{\Phi  _I}=\sum_{I}c_I\ket{\Psi _I}
\end{align}
我们可以得到它能量的导数为
\begin{align}
	\label{C.16}
	\frac{\partial E}{\partial X_A}=\frac{\partial {\tilde{E} }}{\partial X_A}
	+\sum_{\mu i}\frac{\partial E}{\partial C_{\mu i}}\frac{\partial C_{\mu i}}{\partial X_A}
	+\sum_{I}\frac{\partial E}{\partial c_{I}}\frac{\partial c_{I}}{\partial X_A}
\end{align}
式中第一个求和遍历所有的分子轨道系数。在这个情况下,只有1个多组态自洽场(MCSCF)函数时$\frac{\partial E}{\partial X_A}=\frac{\partial {\tilde{E} }}{\partial X_A}$。
对于一般的Hartree-Fock和CI波函数,$\frac{\partial E}{\partial c_{I}}$,则
\begin{align}
	\label{C.17}
	\frac{\partial E}{\partial X_A}=\frac{\partial {\tilde{E} }}{\partial X_A}
	+\sum_{\mu i}\frac{\partial E}{\partial C_{\mu i}}\frac{\partial C_{\mu i}}{\partial X_A}
\end{align}
计算$\frac{\partial C_{\mu i}}{\partial X_A}$很复杂,但是可以通过微扰理论
\endnote{See,for example,J.A.Pople,H.Krishnan,H.B.Schlegel,and J.S.Binkley,
$Int.J.Quantum Chem.\\ \bf{S13:}$225(1979),and refences therein.}解决。
对于一个大的CI系统能量对$\mathbf{C}$的依赖性降低时;或者对于没有大量极性键的系统时;或者对于分子轨道由对称性决定的系统时,第二项对受力的贡献可能很小。
在这种情况下,对于势能面的初始搜索可以使用\autoref{C.9},但是对于精确的结果来说,依赖于这种近似是不令人满意的。

Hartree-Fock能量的二阶导数可以从\autoref{C.12}直接得到
\begin{align}
	\nonumber
	\frac{\partial^2 E}{\partial X_A\partial X_B}=&\sum_{ \mu \nu }P_{\nu \mu}\frac{\partial^2 H_{\mu \nu}^{core}}{\partial X_A\partial X_B}
		+\frac{1}{2}\sum_{\mu \nu \sigma \lambda }P_{ \nu \mu }P_{\lambda \sigma}\frac{\partial^2 (\mu \nu|| \sigma\lambda)}{\partial X_A\partial X_B}
		\\ \nonumber &
		-\sum_{\mu \nu }Q_{ \nu \mu}\frac{\partial^2 S}{\partial X_A\partial X_B}
		+\frac{\partial^2 V_{NN}}{\partial X_A\partial X_B}
		+\sum_{\mu \nu }\frac{\partial P_{ \nu\mu}}{\partial X_B}\frac{\partial H_{\mu \nu}^{core}}{\partial X_A}
		\\ \nonumber &
		+\sum_{\mu \nu \sigma \lambda }\frac{\partial P_{ \nu \mu }}{\partial X_B}P_{\lambda \sigma}\frac{\partial (\mu \nu|| \sigma\lambda)}{\partial X_A}
		-\sum_{\mu \nu }\frac{\partial Q_{ \nu\mu}}{\partial X_B}\frac{\partial S_{\mu \nu}}{\partial X_A}
	\end{align}
表达式的最后三项包含了分子轨道系数的导数并且不能用简单的方式避开。它们可以通过耦合微扰Hartree-Fock理论(CPHF)$^3$得到。
%%%%%
\section{优化技术}
%%%%%
\section{一些优化算法}
%%%%%
\section{过渡态}
%%%%%
\section{约束变分}
%%%%%
\newpage
\theendnotes
\addcontentsline{toc}{section}{注释}
