\chapter{几何结构优化与解析导数法(作者:M.C.Zerner)}
Michael C.Zerner

量子理论项目

佛罗里达大学
\section{引言}

分子构象的再现和预测是分子量子力学的最成功的应用之一。
许多情况下,对于很多简单的分子轨道模型或者最小基组的从头算方法,键长可以重现到$\pm 0.02$\r A,
键角可以重现到$5^{\circ}$。
对于大一点的基组,尤其是那些双$\zeta$加极化的基组;和包含电子相关能的方法,现在生成的几何构型已经
可以达到晶体学精度了。
随着构象计算的日益成功,人们甚至可以选择孤立分子的计算结果,而不是在凝聚介质中获得的实验结果,因为前者可能更适合气相。

除了产生关于势能面全局最小点的信息外,量子力学计算还可以产生局域最小点的信息,这些局域最小点可能不会被直接观测到,但是
很可能会被包哦旱灾反应路径中。类似的,关于过渡态和能垒的信息也可以得到,这些信息一般很难甚至不可能通过其他方式获得。

收集一个势能面上这些所有的信息是很困难的。对于N个原子的体系,它的能量是一个具有$3N-6$(或者$3N-5$)个自由度的函数。
为了进行详细的统计计算,人们可能不得不面对这个“3N”问题,并访问势能面上所有热力学可得的区域。
然而,本附录只涉及势能面的一小部分:那些对应于极小值的点,代表稳态或亚稳态的构象,以及对应于过渡态的点。
%%%%%%%
\section{概论}

在Born-Oppenheimer近似下得到的分子体系的能量$E$是一个以核坐标为参数的函数,记核坐标为
$\mathbf{X}^{\dagger}=(X_1,X_2,\dots,X_{3N})$。
我们希望从$E(\mathbf{X})$进而得到$E(\mathbf{X_1})$,$\mathbf{q=(X_1-X)}$。
我们将能量对$\mathbf{X}$进行泰勒展开:
\begin{align}
	\label{C.1}
	E(\mathbf{X_1})=E(\mathbf{X})+\mathbf{q}^{\dagger}\mathbf{f(X)}
                    +\frac{1}{2}\mathbf{q}^{\dagger}\mathbf{H(X)}\mathbf{q}+\dots
\end{align}
式中梯度为
\begin{align}
	f_i=\frac{\partial E(\mathbf{X})}{\partial X_i}
    \nonumber
\end{align}
Hessian矩阵元为
\begin{align}
	H_{ij}=\frac{\partial E(\mathbf{X})}{\partial X_i\partial X_j}
    \nonumber
\end{align}
注意,列矩阵的下标表示不同的矩阵,如$\mathbf{X_1}$、$\mathbf{X_2}$等等;而$X_i$表示矩阵$\mathbf{X}$第i个元素。
虽然泰勒展开是无穷项的,但是接近极值时,我们希望二阶展开是足够的;例如,对于$\mathbf{X}=\mathbf{X_e}$,其中$\mathbf{X_e}$
表示一个驻点,根据定义$\mathbf{f(X_e)=0}$,则
\begin{align}
	\nonumber
	E(\mathbf{X_1})=E(\mathbf{X_e})+\frac{1}{2}\mathbf{q}^{\dagger}\mathbf{H(X_e)}\mathbf{q}
\end{align}
类似的,
\begin{align}
	\label{C.2}
	\mathbf{f(X_1)}=\mathbf{f(X)}+\mathbf{H(X)}\mathbf{q}
\end{align}
对点$\mathbf{X_1}=\mathbf{X_e}$
\begin{align}
	\label{C.3}
	\mathbf{f(X)}=-\mathbf{H(X)}\mathbf{q}
\end{align}

\autoref{C.3}的解是不显含$\mathbf{X}$的$E(\mathbf{X})$泛函寻找多变量函数极值的最高效方法的初始点。
如果$\mathbf{H}$是非奇异的,则有
\begin{align}
	\label{C.4}
	\mathbf{q}=-\mathbf{H^{-1}(X)}\mathbf{f(X)}
\end{align}
这能够从任意一点$\mathbf{X}$解得$\mathbf{X_e}$,从而使能量函数接近二阶展开。同样的,一个预测的能量$\mathbf{E(H_e)}$
\footnote{译者注:原文如此,此处可能是笔误,结合上下文来看,应为$\mathbf{E(X_e)}$}
可以从下式得到
\begin{align}
	\label{C.5}
	\mathbf{E(X_e)}&=\mathbf{E(X)}-\frac{1}{2}\mathbf{f(X)}^{\dagger}\mathbf{H^{-1}(X)}\mathbf{f(X)}
	\nonumber\\
	&=\mathbf{E(X)}-\frac{1}{2}\mathbf{q}^{\dagger}\mathbf{H(X_e)}\mathbf{q}
\end{align}
对于特殊的势能面寻找极值的问题,我们必须要指出:除非将代表$\mathbf{H}$的零特征值的旋转和平移移除,
否则$\mathbf{H^{-1}(X)}$将不存在。这个工作可以通过Wilson和Eliashevich提出的$\mathbf{B}$矩阵实现:
\endnote{See,for example,E.B.Wilson,J.C.Decius,and P.C.Cross,\textit{Molecular Vibrations},
McGraw-Hill,New York,1955.}
\begin{align}
	\label{C.6}
	\mathbf{Y}^{\dagger}=\mathbf{X}^{\dagger}\mathbf{B}
\end{align}
式中$\mathbf{X}$是$3N$维,$\mathbf{B}$是$3N\times 3(N-6)$维,关联内坐标和原本的笛卡尔坐标$\mathbf{X}$。
揭示这些新坐标下的Taylor级数结构的最简单的方法是考虑功$w$,因为它与坐标系的选择是独立的。
功是力和距离的乘积,能够在任何坐标系下表示为
\begin{align}
	\nonumber
	w=\mathbf{f}^{\dagger}\mathbf{q}=\mathbf{f_y}^{\dagger}\mathbf{q_y}=\mathbf{f_y}^{\dagger}\mathbf{B}^{\dagger}\mathbf{q}
\end{align}
\begin{align}
	\nonumber
	\mathbf{f}^{\dagger}=\mathbf{f_y}^{\dagger}\mathbf{B}^{\dagger}
\end{align}
或者
\begin{align}
	\label{C.7}
	\mathbf{f_y}=\mathbf{f}^{\dagger}(\mathbf{B}^{\dagger})^{-1}
\end{align}
式中$(\mathbf{B}^{\dagger})^{-1}$满足
\begin{align}
	\nonumber
	\mathbf{B}^{\dagger}(\mathbf{B}^{\dagger})^{-1}=\mathbf{1}
\end{align}
下面给出上述方程的一般解为
\begin{align}
	\nonumber
	(\mathbf{B}^{\dagger})^{-1}=\mathbf{mB}(\mathbf{B^{\dagger}mB})^{-1}
\end{align}
式中$\mathbf{m}$是一个任意的$3N\times 3N$矩阵,经常选择一个对角矩阵,
矩阵元素是每一个对应位置上原子的原子质量的倒数的三倍。
也可以选择一个单位矩阵,其中6(或5)个元素选择为0去阻止平动和转动。
这种类型一个简单的选择是将原子1放在原点,原子2在z轴上,原子3在xz平面上。
于是这被移除的6(或5)个坐标为$x_1=y_1=z_1=0$,$x_2=y_2=0$,$y_3=0$。
如果$y_3=0$对于任意选择的第三个原子来说都意味着$x_3=0$,那么这个分子是线性的并且只有5个自由度被选择。
实际操作中,求$\mathbf{H}$的逆矩阵被证明有一点困难。
接下来要讨论的更新的方法直接构建$\mathbf{H}^{-1}$,并且从不更新平动和转动。
在解析求解$\mathbf{H}$的时候,平动和转动可以像之前讨论的那样被移除,或者,如果通过对角化求逆的话,
$\mathbf{H}$的6(或5)个为0的特征值会被一个任意的大数替换掉,本质上在求逆中解耦这些状态。
%%%%%
\section{解析导数}
%%%%%
\section{优化技术}
%%%%%
\section{一些优化算法}
%%%%%
\section{过渡态}
%%%%%
\section{约束变分}
%%%%%
\newpage
\theendnotes
\addcontentsline{toc}{section}{注释}