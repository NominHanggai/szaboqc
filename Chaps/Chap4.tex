\chapter{组态相互作用}
上一章我们讲了\hft 近似, 虽然它在很多情形下非常成功, 但也有本身的局限. \hft 比如在预测$\text{N}_2$电离势的顺序时, 就会发生\emph{定性}的错误. 另外, 限制性HF方法无法正确描述分子解离为开壳层产物的过程(比如$\hd\to2\text{H}$). 虽然非限制HF方法可以定性正确地描述解离, 但给出的势能取现却不精确. 本书剩余部分讲述改进\hft 近似的几种办法. 我们会关心如何获得相关能($E_\mathrm{corr}$), 即精确的非相对论能量($\mathscr{E}_0$)与完备基组极限下\hft 能量($E_0$)的差值.
\begin{equation}
E_\mathrm{corr}=\mathscr{E}_0-E_0
\end{equation}
\section{多组态波函数及Full CI矩阵的结构}
\subsection{中间归一化及相关能的表达式}
\section{双激发CI}
\section{计算示例}
\section{自然轨道与单粒子约化密度矩阵}
目前为止我们所涉及的行列式和组态都由一组正则\hft 轨道构成。由此得到的CI展开收敛很慢。但很清楚的一点是,任意单电子基构成的$N$-电子组态都可以用来做CI计算。因此,我们可以问,能否找到一组单电子基,由该基组得到的CI展开比\hft 基下的CI展开收敛更快?若是如此,我们就能用更少的组态得到相同的结果。由P.-O L\"owdin引入的\emph{自然轨道}就构成这样一组基。

为定义自然轨道,先来考虑$N$电子系统的一阶密度矩阵。给定一归一化波函数$\Phi$, 则$\Phi(\mathbf{x}_1,\ldots,\mathbf{x}_N)\Phi^*(\mathbf{x}_1,\ldots,\mathbf{x}_N)\ddx_1\cdots\ddx_N$就是电子在$\mathbf{x}_1$附近的\underline{\underline{空间-自旋}体积元}$\ddx_1$内,与此同时另外一个电子在$\mathbf{x}_2$附近的体积元$\ddx_2$内(直到N个电子为止)的概率。若仅对$\mathbf{x}_1$附近的$\ddx_1$内由一个电子的概率感兴趣,不关心其他电子在何处,则须对所有其他电子的\underline{空间-自旋}坐标作平均,也即,对$\mathbf{x}_2,\mathbf{x}_3,\ldots,\mathbf{x}_N$积分:
\begin{align}
\rho(\mathbf{x}_1) = N \int\ddx_2\cdots\ddx_N \,\, \Phi(\mathbf{x}_1,\ldots,\mathbf{x}_N)\Phi^*(\mathbf{x}_1,\ldots,\mathbf{x}_N)
\end{align}
$\rho(\mathbf{x}_1)$就称作\underline{N-电子系统}内单个电子的\emph{约化密度函数}。式中的归一化因子$N$是为了使对密度的积分等于总电子数:
\begin{align}
\int\ddx_1\,\,\rho(\mathbf{x_1}) = N
\end{align}
现在将密度函数$\rho(\mathbf{x}_1)$推广为密度矩阵$\gamma(\mathbf{x}_1,\mathbf{x}_1')$, 定义如下;
\begin{align}
\label{4.35}
\gamma(\mathbf{x}_1,\mathbf{x}_1') = N \int\ddx_2\cdots\ddx_N \,\, \Phi(\mathbf{x}_1,\mathbf{x}_2,\ldots,\mathbf{x}_N)\Phi^*(\mathbf{x}_1',\mathbf{x}_2,\ldots,\mathbf{x}_N)
\end{align}
这个矩阵$\gamma(\mathbf{x}_1,\mathbf{x}_1')$依赖于两个连续指标,我们把它称作\emph{一阶约化密度矩阵}或\emph{单电子约化密度矩阵}或者直接叫做\emph{一矩阵}。注意,一矩阵在连续表象下的对角元就是电子密度:
\begin{align}
\gamma(\mathbf{x}_1,\mathbf{x}_1) = \rho(\mathbf{x}_1)
\end{align}
由于$\gamma(\mathbf{x}_1,\mathbf{x}_1')$是两个变量的函数,我们可以将它用正交归一的\hft 自旋轨道$\{\chi_i\}$展开为
\begin{align}
\gamma(\mathbf{x}_1,\mathbf{x}_1') = \sum_{ij} \chi_i(\mathbf{x}_1) \gamma_{ij} \chi_j^*(\mathbf{x}_1)'
\end{align}
其中
\begin{align}
\gamma_{ij} = \int\ddx_1\ddx_1'\,\,\chi_i^*(\mathbf{x}_1)\gamma(\mathbf{x}_1,\mathbf{x}_1')\chi_j(\mathbf{x}_1')
\end{align}
由$\{\gamma_{ij}\}$构成的矩阵$\gamma$是一矩阵在正交归一基$\{\chi_i\}$下的\underline{离散表示}。

\exercise{证明$\gamma$是Hermitian矩阵.
\Next
证明$\mathrm{tr}\,\gamma=N$.
\Next
考虑单电子算符
\begin{align*}
\mathcal{O}_1 = \sum_{i=1}^{N}h(i)
\end{align*}
\begin{enumerate}[a.]
	\item 证明
	\begin{align*}
	\braket{\Phi|\mathcal{O}_1|\Phi} = \int\ddx_1\,\, [h(\mathbf{x}_1 \gamma(\mathbf{x}_1,\mathbf{x}_1')) ]_{\mathbf{x}_1'=\mathbf{x}_1}
	\end{align*}
	其中的记号$[\,\,]_{\mathbf{x}_1'=\mathbf{x}_1}$的意思是,在$h(\mathbf{x}_1)$作用到$\gamma(\mathbf{x}_1,\mathbf{x}_1')$之后,令$\mathbf{x}_1'$等于$\mathbf{x}_1$.
	\item 证明
	\begin{align*}
	\braket{\Phi|\mathcal{O}_1|\Phi} = \mathrm{tr}\,\mathbf{h}\gamma
	\end{align*}
	其中
	\begin{align*}
	h_{ij} = \braket{i|h|j} = \int\ddx_1\,\, \chi^*(\mathbf{x}_1) h(\mathbf{x}_1) \chi_j(\mathbf{x}_1)
	\end{align*}
	如此,单电子算符的期望值就可以用\underline{一矩阵}表示出来。
\end{enumerate}
}

若$\Phi$是\hft 基态波函数,那么可以从\underline{一矩阵}的定义(式\eqref{4.35})证明:
\begin{align}
\gamma^\mathrm{HF}(\mathbf{x}_1,\mathbf{x}_1') = \sum_a \chi_a(\mathbf{x}_1)\chi_a^*(\mathbf{x}_1')
\end{align}
其中的求和仅遍及$\Psi_0$中的自旋轨道。如此,HF\underline{一矩阵}的离散表示非常简单——$\gamma^\mathrm{HF}$是对角的,对角元中,对应占据自旋轨道的元素为一,对应非占轨道的元素为零:
\begin{equation}
\begin{aligned}
\gamma_{ij} & =\delta_{ij}&i,j\in\text{占据轨道}\\
            & = 0 &其他\label{4.40} 
\end{aligned}
\end{equation}
$\gamma^\mathrm{HF}$的对角元可以看作占据数:被占自旋轨道的占据数为1,未占轨道的为0.
\exercise{
回忆在二次量子化中,单电子算符写作:
\begin{align*}
\mathcal{O}_1 = \sum_{ij}\braket{i|h|j}\cs_i a_j
\end{align*}
\begin{enumerate}[a.]
	\item 证明
	\begin{align*}
	\gamma_{ij} = \braket{\Phi|\cs_j a_i|\Phi}
	\end{align*}
	\item 证明$\gamma^\mathrm{HF}$的矩阵元由式\eqref{4.40}给出。
\end{enumerate}
}
\section{多组态自洽场与广义价键法}
\section{截断CI与尺寸一致性问题}